\documentclass[a4paper,12]{article}
\usepackage{amsmath, amsfonts, amssymb, amsthm, bm, graphics, bbm, color}
\usepackage{graphicx}
\usepackage{subfigure}
\usepackage{mathtools}
\usepackage[table]{xcolor}
\usepackage{url}
\usepackage{mathabx}
\usepackage{cancel}

\usepackage{multicol}
\usepackage{float}
\usepackage{caption}
\usepackage{mathalfa}
\usepackage{hyperref}
\usepackage{afterpage}
\usepackage{multirow}
\usepackage[section]{placeins}

\captionsetup{font=footnotesize}
\captionsetup{width=\textwidth}
\DeclareMathAlphabet\mathbfcal{OMS}{cmsy}{b}{n}
\newtheorem{theorem}{Theorem}[section]
\newtheorem{lemma}[theorem]{Lemma}
\newtheorem{corollary}[theorem]{Corollary}
\newtheorem{proposition}[theorem]{Proposition}
\theoremstyle{definition}
\newtheorem{definition}[theorem]{Definition}
\newtheorem{example}[theorem]{Example}
\newtheorem{remark}[theorem]{Remark}


%center table entries
\newcolumntype{P}[1]{>{\centering\arraybackslash}p{#1}}


\newcommand\y{\cellcolor{gray!50}}

\newcommand\p{\cellcolor{gray!25}}

\makeatletter
\newcommand\makebig[2]{%
  \@xp\newcommand\@xp*\csname#1\endcsname{\bBigg@{#2}}%
  \@xp\newcommand\@xp*\csname#1l\endcsname{\@xp\mathopen\csname#1\endcsname}%
  \@xp\newcommand\@xp*\csname#1r\endcsname{\@xp\mathclose\csname#1\endcsname}%
}
\makeatother

\makebig{biggg} {3.0}
\makebig{Biggg} {3.5}
\makebig{bigggg}{4.0}
\makebig{Bigggg}{4.5}
\makebig{biggggg}{5.0}
\makebig{Biggggg}{5.5}
\makebig{bigggggg}{6.0}
\makebig{Bigggggg}{6.5}

\newcommand\bovermat[2]{%
    \makebox[0pt][l]{$\smash{\overbrace{\phantom{%
                    \begin{matrix}#2\end{matrix}}}^{\text{#1}}}$}#2}

\newcommand\bundermat[2]{%
    \makebox[0pt][l]{$\smash{\underbrace{\phantom{%
                    \begin{matrix}#2\end{matrix}}}_{\text{#1}}}$}#2}

\newcommand\partialphantom{\vphantom{\frac{\partial e_{P,M}}{\partial w_{1,1}}}}


\newcommand{\raisesym}[2]{\raisebox{0.5\depth}{$#1\Biggggg \}$}}

% redefine paper size
\setlength{\oddsidemargin}{0in}
\setlength{\textwidth}{6.4in}
\setlength{\topmargin}{-0.5in}
\setlength{\textheight}{9.9in}
\setlength{\headheight}{0in}

%slanted vector symbols
\renewcommand{\vec}[1]{\mbox{\boldmath$#1$}}
\newcommand{\vect}[1]{\boldsymbol{#1}}

%vertical d symbol for integrals
\newcommand{\dif}{\mathrm{d}}
\newcommand{\im}{\mathrm{i}}

\newcommand{\bbR}{\mathbb{R}}
\newcommand{\tr}{\mathrm{tr}}

\newcommand{\vvtheta}{{\bm {\vartheta}}}
\newcommand{\vvphi}{{\bm {\varphi}}}
\newcommand{\vtheta}{{\bm {\theta}}}


\DeclareMathOperator*{\esssup}{ess \, sup}

\begin{document}
\title{Angle measures for an MPT characterisation of a computed toy gun}
%\author{P.D. Ledger}
\date{5th April 2024}
\maketitle

%Barrel has $\sigma_* = 1.45\times 10^6$ S/m and $\mu_r$ is varied ($\mu_r=100$ corresponds to carbon steel) and the receiver has $4.5 \times 10^6$ S/m and $\mu_r=1$ and is a stainless steel.  The barrel is hollow with a cap at one end.  The length of the barrel is 0.2 m, outer and inner radii of the barrel are 0.02 m and 0.01m, respectively. The box representing the receiver is 0.08 m $\times$ 0.01 m $\times$ 0.15 m.  A mesh of 21,132 unstructured tetrahedra, 5,525 prisms and elements of order $p=0,1,2,3,4,5$ are considered. The prisms are chosen to be used as boundary layers in the barrel, which has a smaller skin depth. 
%
%\begin{figure}[h]
%\begin{center}
%\includegraphics[width=0.5\textwidth]{Gun_modelv2_nonsym_StainSt.png}
%\end{center}
%\caption{Illustration of the toy gun. Barrel has $\sigma_* = 1.45\times 10^6$ S/m and $\mu_r$ is varied ($\mu_r=100$ corresponds to carbon steel) and the receiver has $4.5 \times 10^6$ S/m and $\mu_r$ and is a non-magnetic stainless steel.}
%\end{figure}
%
%When the eigenvalue curves  cross each other then this indicates the presence of eigenvalues with algebraic multiplicity greater than one.  The associated eigenvectors for an eigenvalue with algebraic multiplicity two 
% can be arbitrarily chosen as any two vectors that form a basis for the two-dimensional eigenspace.  In numerical computations, the computation of eigenvectors associated with eigenvalues $\lambda_n , \lambda_m$ is problematic when $\lambda_n \to \lambda_m$. 
%
% Angles between the computed eigenvectors using the metrics $d_R$ and $d_F$ are considered and  the approximations $d_E$ and $d_C$ that do not require knowledge of the eigenvectors. Notice the smooth transition of   $d_R$ and $d_F$ in the regions where the eigenvalues are repeated  and the associated {\bf peaks}.
%While we know there are issues with the eigenvectors of repeated eigenvalues, the peaks are interesting as there is a {\bf smooth transition to them}.
%  While $d_E$ and $d_C$ do not capture the peaks well, particularly $d_E$,  the normalising constant involved in the computation do identify the peaks.
%  
%  Note that $d_E$ and $d_C$ do capture the behaviour of $d_R$ and $d_F$ well for other objects when there are no peaks.
%
%
%
%\clearpage
{}
%\subsection{Barrel with $\mu_r=20$}

\begin{figure}[h]
\begin{center}
$\begin{array}{ccc}
\includegraphics[width=0.3\textwidth]{mu20/OCC_Gun_modelv2_nonsym_StainSt_eig_R_al_0.01_20,1_sig_1e6,1e8_ord_5.pdf} &
\includegraphics[width=0.3\textwidth]{mu20/OCC_Gun_modelv2_nonsym_StainSt_eig_I_al_0.01_20,1_sig_1e6,1e8_ord_5.pdf} &
\includegraphics[width=0.3\textwidth]{mu20/OCC_Gun_modelv2_nonsym_StainSt_RI_al_0.01_20,1_sig_1e6,1e8_ord_4.pdf} \\
\lambda_i({\mathcal R}) &\lambda_i({\mathcal I}) &\text{Angles} \\ 
\end{array}$
\end{center}
\caption{Toy gun with $\mu_r=20$ in the barrel.  }
\end{figure}


%\subsection{Barrel with $\mu_r=40$}

\begin{figure}[h]
\begin{center}
$\begin{array}{ccc}
\includegraphics[width=0.3\textwidth]{mu40/OCC_Gun_modelv2_nonsym_StainSt_eig_R_al_0.01_40,1_sig_1e6,1e8_ord_4.pdf} &
\includegraphics[width=0.3\textwidth]{mu40/OCC_Gun_modelv2_nonsym_StainSt_eig_I_al_0.01_40,1_sig_1e6,1e8_ord_4.pdf} &
\includegraphics[width=0.3\textwidth]{mu40/OCC_Gun_modelv2_nonsym_StainSt_RI_al_0.01_40,1_sig_1e6,1e8_ord_4.pdf} \\
\lambda_i({\mathcal R}) &\lambda_i({\mathcal I}) &\text{Angles} \\ 
\end{array}$
\end{center}
\caption{Toy gun with $\mu_r=40$ in the barrel.  }
\end{figure}

%\subsection{Barrel with $\mu_r=80$}

\begin{figure}[h]
\begin{center}
$\begin{array}{ccc}
\includegraphics[width=0.3\textwidth]{mu80/OCC_Gun_modelv2_nonsym_StainSt_eig_R_al_0.01_80,1_sig_1e6,1e8_ord_5.pdf} &
\includegraphics[width=0.3\textwidth]{mu80/OCC_Gun_modelv2_nonsym_StainSt_eig_I_al_0.01_80,1_sig_1e6,1e8_ord_5.pdf} &
\includegraphics[width=0.3\textwidth]{mu80/OCC_Gun_modelv2_nonsym_StainSt_RI_al_0.01_80,1_sig_1e6,1e8_ord_5.pdf} \\
\lambda_i({\mathcal R}) &\lambda_i({\mathcal I}) &\text{Angles} \\ 
\end{array}$
\end{center}
\caption{Toy gun with $\mu_r=80$ in the barrel.  }
\end{figure}

%\subsection{Barrel with $\mu_r=100$}

\begin{figure}[h]
\begin{center}
$\begin{array}{ccc}
\includegraphics[width=0.3\textwidth]{mu100/OCC_Gun_modelv2_nonsym_StainSt_eig_R_al_0.01_100,1_sig_1e6,1e8_ord_5.pdf} &
\includegraphics[width=0.3\textwidth]{mu100/OCC_Gun_modelv2_nonsym_StainSt_eig_I_al_0.01_100,1_sig_1e6,1e8_ord_5.pdf} &
\includegraphics[width=0.3\textwidth]{mu100/OCC_Gun_modelv2_nonsym_StainSt_RI_al_0.01_100,1_sig_1e6,1e8_ord_5.pdf} \\
\lambda_i({\mathcal R}) &\lambda_i({\mathcal I}) &\text{Angles} \\ 
\end{array}$
\end{center}
\caption{Toy gun with $\mu_r=100$ in the barrel.  }
\end{figure}



\end{document}